\begin{rubric}{Misc Experience}
% \subrubric{Awards and Achievements}
% \entry*[2002] \textbf{Merit Award}, Random Training Course held at Secret Location.
% %
% \entry*[2001] \textbf{Department Prize for Outstanding Student Performance}, Unseen University.

\subrubric{Coded Projects -- Open Source Contributions \& Masters Work}

\entry*[Summer 2023] \textbf{RPM Problems Solver}. Developed as part of the Knowledge-Based AI (KBAI) course to create an intelligent agent capable of solving the Ravens IQ Test.  
\textit{Score: 74.00 / 96.00}  

\entry*[Fall 2023] \textbf{\href{https://www.youtube.com/watch?v=BDsmPKvq3rs}{Job Comparison Android App}}. Built a data-persistent Android application to compare and rank jobs based on multiple parameters. Developed during the Software Development Process course.  

\entry*[Spring 2024] \textbf{\href{https://github.com/nkapila6/mlrose-ky}{Machine Learning Projects}}. Implemented ML algorithms, including classification/regression, clustering, dimensionality reduction, and \href{https://www.youtube.com/watch?v=6nc1xtTQcNY}{reinforcement learning agents}. This \href{https://nkapila.me/masters/cs7641-review}{post} on my technical blog details the adventures in this class.

\entry*[Summer 2024] \textbf{\href{https://github.com/nkapila6/lstm-bgd2}{LSTMs on BGD2 dataset}} Created a full end-to-end Machine Learning pipeline using PyTorch \& Metaflow during my research work. Deployed on Docker.

\entry*[Fall 2024] \textbf{Deep Learning Projects}. Developed FCNNs, CNNs, generative models, RNNs, LSTMs, and Transformers using NumPy and PyTorch, Designed and implemented the \href{https://github.com/AttentionSeekers/CNNtention}{CNNtention} project.

\entry*[OSS Contribs] Open Source Contributions to many popular projects.
\begin{itemize}
    \item \textbf{\href{https://github.com/tomlin7/biscuit/}{Biscuit Code Editor}}: Added a \href{https://github.com/tomlin7/biscuit/pull/420}{feature} during my free time as a part of Hackoctober Fest 2024.
    \item \textbf{\href{https://nkapila6.github.io/mlrose-ky/}{MLRose}}: Contributed to the mlose optimization library, a resource used in Georgia Tech's CS7641 Machine Learning class. Wrote the full documentation and added a few features. 
\end{itemize}

\entry*[More] \textbf{Full List} of projects can be viewed on my \href{https://github.com/nkapila6}{Github}.

% \subrubric{Projects}
% \entry*[2023] \textbf{RPM Problems Solver (Summer 2023)}
% \text{Developed as part of the Knowledge-Based AI (KBAI) course, this project involved creating an intelligent agent capable of solving the Ravens IQ Test. Score: 74.00 / 96.00}
% \entry*[2023] \textbf{Job Comparison Android App (Fall 2023)}
% \text{Built a data-persistent Android application designed to compare and rank jobs based on multiple parameters. Developed as part of the Software Development Process course.}
% \entry*[\textbf{Machine Learning Projects (Spring 2024)}]  
% Implemented various ML algorithms, including classification/regression, randomized optimizations, clustering, dimensionality reduction, and autonomous reinforcement learning agents.  
% \begin{itemize}
%     \item Contributed to the \textbf{MLRose Optimization Library}, a resource used in Georgia Tech's CS7641 Machine Learning class.
% \end{itemize}
% \entry*[\textbf{Deep Learning Projects (Fall 2024, ongoing)}]  
% Developed Fully Connected Neural Networks (FCNNs) and Convolutional Neural Networks (CNNs) from scratch using NumPy and PyTorch. Built generative models, interpreted CNN models using Torch Captum, and implemented RNNs, LSTMs, and Transformers.  
% \begin{itemize}
%     \item Designed and implemented \textbf{CNNtention}, augmenting traditional CNNs with attention blocks for enhanced performance.
% \end{itemize}

\subrubric{Certifications -- Listing Recent 3 Only}
% \section*{Certifications}
\entry*[2024] \textbf{\href{https://learn.nvidia.com/certificates?id=0x3zAbT6TfilxUwD_kMgaA}{Fundamentals of Deep Learning}}. Awarded by Nvidia.
\entry*[2024] \textbf{\href{https://www.credly.com/badges/88522d33-bcf6-4aca-ae8c-21a58a68a594/print}{Machine Learning with Python}}. Awarded by IBM.
\entry*[2024] \textbf{\href{https://www.coursera.org/account/accomplishments/verify/W6DPNVH4V88P}{Machine Learning Foundations: A Case Study Approach}}. Awarded by University of Washington.
\entry*[Others] \textbf{Full \& Detailed List} can be viewed at my \href{https://www.linkedin.com/in/nikhilkapila/details/certifications/}{LinkedIn}
\end{rubric}